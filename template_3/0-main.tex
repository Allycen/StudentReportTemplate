\documentclass[utf8x, 12pt]{G7-32} 
\include{1-config}


\begin{document}

\frontmatter 
\begin{center} 

\large САНКТ-ПЕТЕРБУРГСИЙ ГОСУДАРСТВЕННЫЙ ПОЛИТЕХНИЧЕСКИЙ УНИВЕРСИТЕТ

\large Кафедра Компьютерных Систем и Программных Технологий \\[5.5cm] 

\huge ОТЧЕТ \\[0.6cm] % название работы, затем отступ 0,6см
\large индивидуальному заданию\\
\large Дисциплина: <<Логическое проектирование>>\\[3.7cm]

\end{center} 

\begin{flushright}
Выполнил: студент гр. 53501/2 \\
Пономарев М.A \\[1.2cm]


Преподаватель \\
Мараховский В.Б
\end{flushright}


\vfill 

\begin{center} 
\large Санкт-Петербург \\
2015
\end{center} 

\thispagestyle{empty}

\thispagestyle{empty}
\setcounter{page}{0}
\tableofcontents
\clearpage
\mainmatter



\chapter{Задание}

\begin{itemize}
	\item На языке Verilog описать представленную ниже схему
	
\begin{figure}[h]
	\begin{center}
		\includegraphics[width=7cm]{img/shema}
	\end{center}
	\vspace{-5mm}\caption{Схема устройства}
\end{figure}	
	
	\item Посмотреть синтезированную пакетом QII схему (RTL Viewer)
	\item Осуществить функциональное моделирование 
	\item Назначить выводы СБИС
	\item Осуществить полную компиляцию, программирование платы и проверить работу проекта на плате 
\end{itemize}

\chapter{Решение}

Для определения логики работы файла создадим Verilog HDL файл. В результате получаем следующий код:

\begin{figure}[hhh!]
	\begin{center}
		\includegraphics[width=12cm]{img/vhdl}
	\end{center}
	\vspace{-5mm}\caption{VHDL код}
\end{figure}


Для просмотра логической схемы работы устройства воспользуемся RTL Viewer. В результате получаем:

\begin{figure}[hhh!]
	\begin{center}
		\includegraphics[width=12cm]{img/RTL_Viewer}
	\end{center}
	\vspace{-5mm}\caption{Логическая схема устройства}
\end{figure}


Осуществим функциональное моделирование. Для этого создадим Vector Waveform File, определим значение входных сигналов. Запустим функциональное моделирование, получим:

\begin{figure}[hhh!]
	\begin{center}
		\includegraphics[width=12cm]{img/waveform}
	\end{center}
	\vspace{-5mm}\caption{Векторная диаграмма}
\end{figure}


\chapter{Выводы}

В ходе выполнения лаобораторной работы было спроектировано устройство на языке Verilog HDL, осуществлено функциональное моделировние для проверки правильности работы и проверка работы устройстройства на плате.


\end{document}
